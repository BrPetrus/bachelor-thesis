\documentclass[
  digital,     %% The `digital` option enables the default options for the
               %% digital version of a document. Replace with `printed`
               %% to enable the default options for the printed version
               %% of a document.
%%  color,       %% Uncomment these lines (by removing the %% at the
%%               %% beginning) to use color in the printed version of your
%%               %% document
  oneside,     %% The `oneside` option enables one-sided typesetting,
               %% which is preferred if you are only going to submit a
               %% digital version of your thesis. Replace with `twoside`
               %% for double-sided typesetting if you are planning to
               %% also print your thesis. For double-sided typesetting,
               %% use at least 120 g/m² paper to prevent show-through.
  nosansbold,  %% The `nosansbold` option prevents the use of the
               %% sans-serif type face for bold text. Replace with
               %% `sansbold` to use sans-serif type face for bold text.
  nocolorbold, %% The `nocolorbold` option disables the usage of the
               %% blue color for bold text, instead using black. Replace
               %% with `colorbold` to use blue for bold text.
  lof,         %% The `lof` option prints the List of Figures. Replace
               %% with `nolof` to hide the List of Figures.
  lot,         %% The `lot` option prints the List of Tables. Replace
               %% with `nolot` to hide the List of Tables.
]{fithesis4}
%% The following section sets up the locales used in the thesis.
\usepackage[resetfonts]{cmap} %% We need to load the T2A font encoding
\usepackage[T1,T2A]{fontenc}  %% to use the Cyrillic fonts with Russian texts.
\usepackage[
  main=english, %% By using `czech` or `slovak` as the main locale
                %% instead of `english`, you can typeset the thesis
                %% in either Czech or Slovak, respectively.
  english, czech, slovak %% The additional keys allow
]{babel}        %% foreign texts to be typeset as follows:
%%
%%   \begin{otherlanguage}{german}  ... \end{otherlanguage}
%%   \begin{otherlanguage}{russian} ... \end{otherlanguage}
%%   \begin{otherlanguage}{czech}   ... \end{otherlanguage}
%%   \begin{otherlanguage}{slovak}  ... \end{otherlanguage}
%%
%% For non-Latin scripts, it may be necessary to load additional
%% fonts:
\usepackage{paratype}
\def\textrussian#1{{\usefont{T2A}{PTSerif-TLF}{m}{rm}#1}}
%%
%% The following section sets up the metadata of the thesis.
\thesissetup{
    date        = \the\year/\the\month/\the\day,
    university  = mu,
    faculty     = fi,
    type        = bc,
    department  = Department of Visual Computing,
    author      = Bruno Petrus,
    gender      = m,
    advisor     = {doc. RNDr. Martin Maška, Ph.D.},
    title       = {Segmentation of Membrane-Stained Cells in Image Data of Organoids},
    TeXtitle    = {Segmentation of Membrane-Stained Cells in Image Data of Organoids},
    keywords    = {keyword1, keyword2, ...},
    TeXkeywords = {keyword1, keyword2, \ldots},
    abstract    = {%
      This is the abstract of my thesis, which can

      span multiple paragraphs.
    },
    thanks      = {%
      These are the acknowledgements for my thesis, which can

      span multiple paragraphs.
    },
    bib         = bibliography.bib,
    %% Remove the following line to use the JVS 2018 faculty logo.
    facultyLogo = fithesis-fi,
}
\usepackage{makeidx}      %% The `makeidx` package contains
\makeindex                %% helper commands for index typesetting.
%% These additional packages are used within the document:
\usepackage{paralist} %% Compact list environments
\usepackage{amsmath}  %% Mathematics
\usepackage{amsthm}
\usepackage{amsfonts}
\usepackage{url}      %% Hyperlinks
\usepackage{markdown} %% Lightweight markup
\usepackage{listings} %% Source code highlighting
\lstset{
  basicstyle      = \ttfamily,
  identifierstyle = \color{black},
  keywordstyle    = \color{blue},
  keywordstyle    = {[2]\color{cyan}},
  keywordstyle    = {[3]\color{olive}},
  stringstyle     = \color{teal},
  commentstyle    = \itshape\color{magenta},
  breaklines      = true,
}
\usepackage{floatrow} %% Putting captions above tables
\floatsetup[table]{capposition=top}
\usepackage[babel]{csquotes} %% Context-sensitive quotation marks

%% Specify new commands
\newcommand*{\R}{\ensuremath{\mathbb{R}}}

\begin{document}
%% The \chapter* command can be used to produce unnumbered chapters:
\chapter*{Introduction}
%% Unlike \chapter, \chapter* does not update the headings and does not
%% enter the chapter to the table of contents. I we want correct
%% headings and a table of contents entry, we must add them manually:
\markright{\textsc{Introduction}}
\addcontentsline{toc}{chapter}{Introduction}

Theses are rumoured to be \enquote{the capstones of education}, so
I decided to write one of my own. If all goes well, I will soon
have a diploma under my belt. Wish me luck!

\begin{otherlanguage}{czech}
Říká se, že závěrečné práce jsou \enquote{vyvrcholením studia}
a tak jsem se rozhodl jednu také napsat. Pokud vše půjde podle
plánu, odnesu si na konci semestru diplom. Držte mi palce!
\end{otherlanguage}

\begin{otherlanguage}{slovak}
Hovorí sa, že záverečné práce sú \enquote{vyvrcholením štúdia}
a tak som sa rozhodol jednu tiež napísať. Ak všetko pôjde podľa
plánu, odnesiem si na konci semestra diplom. Držte mi palce!
\end{otherlanguage}

\parencite{gonzalez2002}

\chapter{Theory}

\section{Digital Image}

Our eyesight is arguably one of our most useful sense through which we look at
the world. We begin by formalising the concept of an image, so that we can use
mathematics and computer science to analyze how one can vyuzit images.
Essentially we can think of an image as an n-dimensional function $f(x_1, x_2,
..., x_n)$, where $x_1, x_2, .. x_n$ are coordinates inside a spatial plane,
while the function value at those coordinates specify the magnitude or intensity
of the signal at that point. In most general term, we can think about images as
a functions  $f:\R^n \rightarrow \R^m$, where $n$ indicates the number of
spatial dimensions and $m$ specifies the number of channels. For example when
talking about two-dimensional grayscale images, $m$ is equal to 1 and $n$ is
equal to 2. 

In real life, we do not necessarily work with real valued spatial dimensions and
intensities, but only have a finite number of bits to work with. The process of
acquiring images in a finite grid and assigning a intensities from a finite
range is called sampling and quantization. In essence, we create a discretized
version of the original signal, which can be represented as an array of values.
This is what we call a digital image.


\subsection{Acquiring}

\subsection{Sampling}

\subsection{Noise}



\section{Edge finding}

\section{Thresholding}

\section{Mathematical morphology}

\section{Segmentation}



\end{document}
